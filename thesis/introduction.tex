% --- Introduction ---
\chapter*{Вступ}
\addcontentsline{toc}{chapter}{Вступ}
\label{ch:introduction}

Актуальність теми магістерської роботи зумовлена зростаючою потребою у спеціалізованих онлайн-платформах для нішевих спільнот, зокрема для бджолярів, та визначається низкою ключових факторів. По-перше, бджільництво відіграє незамінну роль не лише як галузь сільського господарства, що забезпечує виробництво меду, воску, прополісу та інших цінних продуктів, але й як фундаментальний елемент підтримки біорізноманіття та екологічної стабільності через запилення ентомофільних культур та дикорослих рослин. Варто зазначити, що Україна посідає провідні позиції у світовому бджільництві, традиційно входячи до трійки світових лідерів та посідаючи перше місце за обсягами виробництва меду серед країн Європи \cite{unian2015honeyRank}, що підкреслює стратегічну важливість цієї галузі для національної економіки.
Сучасні підходи, такі як впровадження машинного навчання \cite{volohovich2024machinelearning} та використання даних супутникового моніторингу \cite{volohovich2024satellite}, відкривають нові можливості для оптимізації бджільницьких процесів та підвищення ефективності галузі в цілому.
За оцінками експертів, близько третини продовольства, що споживається людством, залежить від запилення комахами, серед яких бджоли є одними з найефективніших.

По-друге, сучасні бджолярі в Україні стикаються зі значними викликами, що загрожують як окремим пасікам, так і галузі в цілому. За два роки повномасштабної війни (2022-2023 рр.) чисельність бджолосімей в країні скоротилася на 13,8\%, до 2,31 млн, порівняно з 2,69 млн на початок 2022 року, при цьому значна частина (близько 30\%) довоєнних пасік опинилася на окупованих територіях \cite{skilky2025beeMortality}. Економічні збитки лише від скорочення кількості бджолосімей станом на кінець 2023 року оцінювалися у 31,9 млн доларів США. Додатково, у 2024 році на контрольованій території України смертність бджіл зросла до критичних 20-25\%, що частково пояснюється несприятливими погодними умовами та скороченням площ медоносних рослин \cite{skilky2025beeMortality}. До цих факторів додаються постійні проблеми з поширенням хвороб та шкідників бджіл, а також масове використання пестицидів у сільському господарстві. В таких складних умовах оперативний обмін достовірною інформацією, передовим досвідом щодо профілактики та лікування хвороб, адаптації до кліматичних та воєнних викликів, а також безпечного співіснування з аграрним виробництвом набуває критичного значення.

По-третє, існуючі загальні соціальні мережі, месенджери та форуми, хоча й використовуються бджолярами для спілкування, не завжди враховують специфічні потреби та унікальний контекст бджолярської спільноти. Вони не надають спеціалізованих інструментів для обговорення вузькопрофільних тем, таких як ветеринарія бджіл, селекція, технології догляду за бджолосім\'ями, особливості медоносних рослин у конкретних регіонах. Відсутні також ефективні механізми для прямої комунікації між бджолярами та агровиробниками щодо графіків обробки полів пестицидами, що є однією з головних причин масової загибелі бджіл. Особливо гострою є проблема координації дій щодо попередження отруєнь бджіл та ефективного планування розташування пасік, особливо зважаючи на те, що галузь переважно (95\%) представлена домогосподарствами, які часто не реєструють пасіки, що ускладнює збір достовірної статистики та координацію \cite{skilky2025beeMortality}.

Таким чином, розробка спеціалізованої веб-платформи, що інтегрує засоби комунікації, базу знань, інструменти для обміну оперативною інформацією (зокрема, про обробки полів та сприяння діалогу з фермерами) та можливості для ведення обліку та оптимального розміщення пасік, є надзвичайно актуальним завданням. Така платформа може суттєво сприяти підвищенню ефективності бджільництва в умовах кризи, збереженню бджолосімей, покращенню координації між пасічниками та іншими зацікавленими сторонами, а також слугувати інструментом для збору більш точних даних, важливих для розвитку галузі та моніторингу екологічного стану.

Метою даної магістерської роботи є розробка фулстек веб-застосунку – веб-платформи для комунікації та обміну знаннями в спільноті бджолярів \textit{Beekeepers Community Platform}, що надасть бджолярам зручні інструменти для комунікації, обміну інформацією, доступу до бази знань та управління даними про власні пасіки та сільськогосподарські угіддя.

Для досягнення поставленої мети було визначено наступні завдання:
\begin{itemize}
    \item Провести детальний аналіз предметної області бджільництва, виявити ключові потреби цільової аудиторії та дослідити наявні на ринку аналоги та спеціалізовані програмні рішення для спільнот.
    \item Обґрунтувати вибір сучасного та ефективного технологічного стеку для розробки повнофункціонального веб-застосунку, враховуючи вимоги до масштабованості, продуктивності та зручності розробки клієнтської (React, TypeScript, Material-UI, Vite) та серверної (NestJS, Fastify, MongoDB) частин.
    \item Спроектувати комплексну архітектуру системи, включаючи:
        \begin{itemize}
            \item Архітектуру клієнтської частини з використанням компонентного підходу, управління станом (наприклад, Redux Toolkit Query (RTK Query), React Context Application Programming Interface (Context API)) та адаптивного дизайну.
            \item Модульну архітектуру серверної частини на базі NestJS, що забезпечує чітке розділення функціональних блоків (автентифікація, управління користувачами, форум, карта, база знань).
            \item Логічну та фізичну схему бази даних MongoDB для ефективного зберігання та обробки користувацьких даних, геопросторової інформації, контенту форуму та бази знань.
        \end{itemize}
    \item Реалізувати ключовий функціонал платформи, зокрема:
        \begin{itemize}
            \item Систему реєстрації, автентифікації (локальна та через Google OAuth 2.0) та авторизації користувачів, включаючи механізм верифікації електронної пошти.
            \item Модуль форуму для створення тем, публікації повідомлень та коментування.
            \item Модуль інтерактивної карти (на базі Leaflet) для додавання, перегляду, редагування та видалення інформації про вулики та сільськогосподарські поля, з візуалізацією дат обробки полів.
            \item Прототип модуля бази знань з можливістю інтеграції AI-асистента (штучний інтелект) для відповідей на поширені питання.
            \item Базовий функціонал адміністрування користувачів.
        \end{itemize}
    \item Імплементувати надійні механізми безпеки, включаючи хешування паролів, захист кінцевих точок прикладного програмного інтерфейсу (API-ендпоінтів) (guards), валідацію вхідних даних на сервері (об'єкти передачі даних – Data Transfer Objects (DTOs)), та налаштування механізму обміну ресурсами між різними джерелами (Cross-Origin Resource Sharing (CORS)).
    \item Розробити інтуїтивно зрозумілий, адаптивний та естетично привабливий користувацький інтерфейс (User Interface, UI), що забезпечує зручну взаємодію (User Experience, UX) з платформою на різних пристроях, використовуючи компоненти Material-UI та принципи UX/UI дизайну.
    \item Провести функціональне тестування розробленого прототипу для перевірки коректності роботи основного функціоналу та підготувати платформу до можливого розгортання.
\end{itemize}

Об'єктом дослідження є процес проектування та розробки веб-платформи для комунікації та обміну знаннями в спільноті бджолярів, спрямованої на вирішення проблем взаємодії з агровиробниками та оптимізації розміщення пасік.

Предметом дослідження є архітектурні рішення, технології, інструменти та методи обробки й аналізу даних (включаючи геопросторові та текстові), що застосовуються для створення інтерактивної та функціональної веб-платформи для бджолярів, спрямованої на вирішення проблем комунікації, ефективного розміщення пасік, обміну знаннями та попередження ризиків.

Методи дослідження, що використовувалися в роботі, включають:
\begin{itemize}
    \item Аналіз науково-технічної літератури та існуючих аналогів.
    \item Системний аналіз та проектування програмного забезпечення.
    \item Об'єктно-орієнтоване програмування.
    \item Використання сучасних веб-технологій та фреймворків: React для розробки клієнтської частини, NestJS (Node.js) для серверної частини, MongoDB як система управління базами даних.
    \item Застосування бібліотек Material-UI для користувацького інтерфейсу та Leaflet для реалізації картографічного функціоналу.
\end{itemize}

\subsection*{Наукова новизна}
Наукова новизна одержаних результатів полягає у наступному:
\begin{itemize}
    \item Розроблено архітектуру та реалізовано функціональний прототип інтегрованої веб-платформи, що забезпечує синергетичне поєднання засобів соціальної комунікації (форум), спеціалізованої бази знань та інструментів геоінформаційного менеджменту (інтерактивна карта пасік та полів) для нішевої спільноти бджолярів. На відміну від розрізнених загальних інструментів чи окремих геоінформаційних систем (ГІС), запропоноване комплексне рішення цілеспрямовано адаптоване до специфічних операційних та інформаційних потреб бджолярів України.
    \item Запропоновано та впроваджено оригінальний метод візуалізації на інтерактивній карті критично важливої для бджолярів інформації про заплановані дати та ризики обробки сільськогосподарських полів. Цей метод, що базується на динамічному кольоровому кодуванні полігонів відповідно до часової близькості обробок, є новим інструментом для підвищення ситуаційної обізнаності бджолярів та оперативного попередження потенційних загроз для бджолосімей у локальному контексті.
    \item Досліджено та обґрунтовано ефективність застосування сучасного, але збалансованого технологічного стеку (React, NestJS, Leaflet, MongoDB) для створення масштабованої та функціональної платформи, що задовольняє специфічні вимоги до розробки систем підтримки агро-геоінформаційних спільнот. Продемонстровано практичну реалізацію такого стеку для вирішення завдань у сфері бджільництва, спираючись на аналіз існуючих підходів до цифрової трансформації сільського господарства та управління онлайн-спільнотами \cite{preece2005onlinecommunities, huet2022digitalbeekeeping, guruprasad2024beeopen}.
    \item Розроблено та інтегровано прототип інтелектуального FAQ-асистента на основі великої мовної моделі (Large Language Model, LLM), з адаптованим промпт-інжинірингом для надання контекстно-залежних відповідей на питання користувачів у специфічній галузі бджільництва, що є новим підходом до надання інформаційної підтримки в рамках подібних нішевих платформ.
\end{itemize}

Практичне значення отриманих результатів полягає у створенні готового до використання прототипу веб-платформи \textit{Beekeepers Community Platform}, що може бути впроваджена для підтримки спільноти бджолярів, покращення їх взаємодії та доступу до актуальної інформації.

\subsection*{Апробація результатів роботи}
Результати роботи були представлені у вигляді тез доповіді на конференції "Формування сучасної науки: методика та практика", 23.05.2025, м. Київ.

Структура роботи: Магістерська робота складається зі вступу, пʼяти розділів, висновків, списку використаних джерел та додатків. 