\section*{Висновки та рекомендації}
\addcontentsline{toc}{section}{Висновки та рекомендації}
\label{sec:conclusions_recommendations}

У результаті проходження науково-дослідницької практики та на основі розробки веб-застосунку \textit{Beekeepers Community Platform} було проаналізовано процес створення сучасного IT-продукту, від ідеї до базової реалізації та розгортання. Ключові технології, що використовувались, включали фреймворк React \cite{react} для фронтенду та NestJS \cite{nestjs} для бекенду, що дозволило створити модульну та масштабовану архітектуру.

Було розглянуто основні етапи розробки:
\begin{itemize}
    \item Аналіз предметної області та визначення вимог користувачів.
    \item Проєктування архітектури системи, включаючи базу даних (використовувалася MongoDB \cite{mongodb}) та API.
    \item Реалізація основного функціоналу: управління пасіками, події календаря, форум для спільноти, інтерактивна карта (реалізована за допомогою Leaflet \cite{leaflet}).
    \item Налаштування середовища для розробки та розгортання за допомогою Docker.
\end{itemize}

Практика дозволила закріпити теоретичні знання та отримати практичний досвід роботи з сучасними інструментами та методологіями розробки програмного забезпечення. Було досліджено важливість інтеграції різних компонентів системи та забезпечення їх взаємодії.

Рекомендації для подальшого дослідження та розвитку платформи \textit{Beekeepers Community Platform}:
\begin{itemize}
    \item Впровадження модуля аналітики стану бджолиних сімей на основі даних користувачів.
    \item Розширення функціоналу інтерактивної карти: додавання фільтрів за типами культур/станом цвітіння, можливість позначення зон обробки хімікатами.
    \item Інтеграція з зовнішніми сервісами (наприклад, погодними API) для надання корисних рекомендацій бджолярам.
    \item Покращення користувацького інтерфейсу та досвіду користувача (UI/UX).
\end{itemize}

Загалом, науково-дослідницька практика надала цінний досвід у сфері аналізу та розробки IT-продуктів.

Отримані знання та практичні навички створили міцне підґрунтя для подальших досліджень у галузі розробки веб-застосунків, аналітики даних та управління IT-проектами.
\newpage 