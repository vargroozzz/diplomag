% --- Chapter 2: Platform Design ---
\chapter{Проектування веб-платформи}
\label{ch:design}

\section{Аналіз вимог до веб-платформи для спільноти бджолярів}
\label{sec:requirements}
Проектування веб-платформи для комунікації та обміну знаннями в спільноті бджолярів \textit{Beekeepers Community Platform} базувалося на визначенні ключових функціональних та нефункціональних вимог, що забезпечують його корисність, надійність та зручність для користувачів.

\subsection{Функціональні вимоги}
\begin{itemize}
    \item \textbf{Реєстрація та автентифікація користувачів:} Можливість створення облікового запису з використанням електронної пошти та паролю, верифікація email через надсилання підтверджувального листа (токен дійсний 1 годину), можливість повторного надсилання листа верифікації, а також автентифікація за допомогою облікового запису Google (OAuth 2.0).
    \item \textbf{Управління профілем користувача:} Перегляд та редагування базової інформації профілю (наприклад, біографія, місцезнаходження, експертиза).
    \item \textbf{Форум для обговорень:} Створення нових тем для обговорення, публікація повідомлень у темах, можливість залишати коментарі до повідомлень, система вподобань (лайків) для постів.
    \item \textbf{База знань:} Доступ до каталогу статей та ресурсів з бджільництва, можливість пошуку та фільтрації матеріалів за категоріями (поточна реалізація з mock-даними).
    \item \textbf{Інтерактивна карта (Управління пасіками та полями):} 
        \begin{itemize}
            \item Відображення карти (Україна за замовчуванням) з використанням Leaflet.
            \item Додавання точкових маркерів для вуликів із зазначенням назви, нотаток та автоматичним визначенням геолокації.
            \item Відображення вуликів з кастомними іконками (MUI HiveIcon).
            \item Видалення маркерів вуликів з картографічного інтерфейсу (з підтвердженням).
            \item Додавання полігональних об'єктів для полів із зазначенням назви, типу культури, періоду цвітіння та списку запланованих дат обробки.
            \item Редагування метаданих існуючих полів (назва, тип культури, період цвітіння, дати обробки).
            \item Динамічне візуальне виділення полів на карті різними кольорами (наприклад, червоний, помаранчевий, синій) залежно від статусу та терміновості запланованих обробок.
            \item Відображення детальної інформації (метаданих) при виборі об'єкта на карті у спливаючих вікнах (Popups).
            \item Фільтрація об'єктів на карті за різними критеріями (тип культури, період цвітіння для полів; тип вулика, стан для пасік) -- (заплановано/майбутній функціонал).
            \item Можливість отримання сповіщень про заплановані обробки полів поблизу пасік (майбутній функціонал).
        \end{itemize}
    \item \textbf{Адміністрування користувачів (для ролі Адміністратор):}
        \begin{itemize}
            \item Перегляд списку всіх зареєстрованих користувачів системи.
            \item Можливість зміни ролі користувача (надання/скасування прав адміністратора).
        \end{itemize}
    \item \textbf{Інтелектуальний помічник FAQ (на базі штучного інтелекту, ШІ):}
        \begin{itemize}
            \item Можливість для користувачів ставити питання природною мовою стосовно функціоналу платформи або загальних тем бджільництва (обмежено наданим контекстом).
            \item Отримання відповідей, згенерованих моделлю OpenAI (наприклад, GPT-3.5-turbo) на основі попередньо визначеного набору Часто Задаваних Питань (ЧаПи, FAQ) або даних з бази знань.
            \item Інтерфейс для введення питання та відображення відповіді.
        \end{itemize}
    \item \textbf{Інтернаціоналізація:} Підтримка декількох мов інтерфейсу (українська, англійська).
\end{itemize}

\subsection{Нефункціональні вимоги}
\begin{itemize}
    \item \textbf{Продуктивність:} Забезпечення прийнятного часу завантаження сторінок та швидкої відповіді сервера на запити користувачів.
    \item \textbf{Безпека:} Захист облікових записів користувачів (хешування паролів), валідація вхідних даних на клієнті та сервері, використання захищеного протоколу передачі гіпертексту (HTTPS) у продакшн-середовищі, захист від поширених веб-вразливостей (наприклад, міжсайтовий скриптинг (Cross-Site Scripting, XSS) через використання React, який екранує дані за замовчуванням; підробка міжсайтових запитів (Cross-Site Request Forgery, CSRF) – потребує відповідних механізмів захисту).
    \item \textbf{Масштабованість:} Архітектура застосунку повинна дозволяти майбутнє масштабування для обслуговування зростаючої кількості користувачів та обсягів даних.
    \item \textbf{Надійність:} Система повинна бути доступною та стабільно працювати.
    \item \textbf{Зручність використання (Usability):} Інтерфейс має бути інтуїтивно зрозумілим, легким у навігації та адаптивним для різних розмірів екранів (десктоп, мобільні пристрої).
    \item \textbf{Підтримуваність коду:} Кодова база повинна бути добре структурованою, документованою (де необхідно) та легкою для модифікації та розширення.
\end{itemize}

\section{Аналіз існуючих рішень та обґрунтування розробки}
\label{sec:existing_solutions}
Перед тим як приступити до проектування веб-платформи для комунікації та обміну знаннями в спільноті бджолярів \textit{Beekeepers Community Platform}, було проведено аналіз існуючих рішень, які частково або повністю можуть задовольняти потреби бджолярської спільноти. Цей аналіз допоміг виявити незаповнені ніші та обґрунтувати необхідність створення нового спеціалізованого інструменту.

\subsection{Загальноцільові платформи та соціальні мережі}
Багато бджолярів активно використовують популярні соціальні мережі (наприклад, Facebook групи), загальні форуми та месенджери (Telegram, Viber) для спілкування, обміну досвідом та швидкого отримання відповідей на нагальні питання. 

\textbf{Переваги:}
\begin{itemize}
    \item Велика аудиторія та низький поріг входження.
    \item Швидкість поширення інформації та можливість миттєвого зворотного зв\'язку.
    \item Безкоштовність використання.
\end{itemize}

\textbf{Недоліки:}
\begin{itemize}
    \item \textbf{Відсутність спеціалізованого функціоналу:} Загальні платформи не надають інструментів, специфічних для бджільництва, таких як інтерактивні карти для позначення пасік та полів, системи сповіщень про обробку полів, структуровані бази знань з ветеринарії бджіл чи агротехніки медоносних культур.
    \item \textbf{Низька структурованість інформації:} Важлива інформація часто губиться у потоці повідомлень, її важко знайти та систематизувати. Пошук по історії обговорень є неефективним.
    \item \textbf{Відволікаючі фактори та нерелевантний контент:} Великий обсяг сторонньої інформації та реклами знижує концентрацію на професійних питаннях.
    \item \textbf{Проблеми з достовірністю інформації:} Відсутність модерації з боку експертів у бджільництві може призводити до поширення неперевірених порад та рекомендацій.
    \item \textbf{Обмежені можливості для геолокаційних сервісів:} Координація розташування пасік, моніторинг медоносної бази або відстеження зон обробки полів практично неможливі на таких платформах.
\end{itemize}

\subsection{Спеціалізовані ресурси та форуми для бджолярів (закордонні та вітчизняні)}
На ринку існує певна кількість спеціалізованих веб-ресурсів, форумів та порталів, присвячених бджільництву. 
% TODO: Add examples of specific existing platforms if known, e.g.:
% - BeekeepingForum.com (міжнародний, загальний)
% - Toito.com.ua (український, оголошення та статті)
% - Regional beekeeping association sites
% For each, briefly discuss its focus, strengths, and weaknesses.

\textbf{Загальні характеристики та потенційні недоліки:}
\begin{itemize}
    \item Багато існуючих форумів мають застарілий дизайн та обмежений функціонал, що не відповідає сучасним вимогам до користувацького досвіду.
    \item Часто відсутня інтеграція з картографічними сервісами або вона є рудиментарною.
    \item Монетизація через велику кількість реклами може погіршувати зручність використання.
    \item Локальні українські ресурси можуть мати обмежену аудиторію або бути недостатньо активними.
    \item Відсутність єдиної платформи, що комплексно поєднує форум, базу знань, карту та інструменти сповіщення, змушує бджолярів використовувати декілька розрізнених інструментів.
\end{itemize}

\subsection{Обґрунтування необхідності розробки платформи}
Аналіз існуючих рішень показує, що незважаючи на наявність загальних платформ для спілкування та деяких спеціалізованих ресурсів, існує значна потреба у комплексній, сучасній та функціональній веб-платформі для української спільноти бджолярів. Веб-платформа для комунікації та обміну знаннями в спільноті бджолярів \textit{Beekeepers Community Platform} покликана заповнити цю нішу, надаючи наступні ключові переваги:
\begin{itemize}
    \item \textbf{Інтегрований підхід:} Поєднання форуму, бази знань, інтерактивної карти з функціями управління пасіками/полями та сповіщеннями про обробки в єдиному інтерфейсі.
    \item \textbf{Спеціалізація на потребах бджолярів:} Функціонал, розроблений з урахуванням специфіки галузі (наприклад, візуалізація дат обробки полів, типи даних для профілю бджоляра).
    \item \textbf{Сучасний користувацький досвід:} Інтуїтивно зрозумілий, адаптивний дизайн та швидка робота застосунку.
    \item \textbf{Потенціал для локалізації та розвитку:} Можливість адаптації до регіональних особливостей України та подальшого розширення функціоналу (наприклад, інтеграція з AI-асистентом, ринок продукції бджільництва тощо).
    \item \textbf{Сприяння формуванню сильної спільноти:} Централізована платформа може стати осередком для обміну знаннями, координації дій та вирішення спільних проблем бджолярів.
\end{itemize}
Таким чином, розробка запропонованої платформи є актуальною та обґрунтованою, оскільки вона спрямована на вирішення реальних проблем цільової аудиторії та пропонує унікальний набір інструментів, відсутній у більшості існуючих альтернатив.

\section{Діаграми варіантів використання (Use Case Diagrams)}
\label{sec:use_cases}
% TODO: Create Use Case diagrams for major actors (Unregistered User, Registered User, Admin (if any)).
% Example:
% \begin{figure}[H]
%   \centering
%   \includegraphics[width=0.8\textwidth]{images/use_case_diagram.png} % Replace with your actual diagram
%   \caption{Діаграма варіантів використання для зареєстрованого користувача}
%   \label{fig:use_case_registered_user}
% \end{figure}

На діаграмах варіантів використання візуалізуються основні актори системи та їх взаємодія з функціональними можливостями платформи. В рамках веб-платформи для комунікації та обміну знаннями в спільноті бджолярів \textit{Beekeepers Community Platform} виділено три основні категорії акторів:

\subsection{Незареєстрований користувач}
Незареєстрований користувач має обмежений доступ до функціоналу платформи. Основні варіанти використання:
\begin{itemize}
    \item Перегляд публічної інформації (головна сторінка, загальний опис платформи)
    \item Реєстрація нового облікового запису
    \item Вхід до системи (якщо вже має обліковий запис)
\end{itemize}

\begin{figure}[htbp]
  \centering
  \includegraphics[width=0.7\textwidth]{practice_report/images/usecase_guest.png}
  \caption{Діаграма варіантів використання для незареєстрованого користувача}
  \label{fig:usecase_guest}
\end{figure}

\subsection{Зареєстрований користувач}
Після автентифікації користувач отримує доступ до основного функціоналу платформи:
\begin{itemize}
    \item \textbf{Управління профілем:}
    \begin{itemize}
        \item Редагування особистої інформації
        \item Зміна паролю
        \item Перегляд своєї активності на платформі
    \end{itemize}
    
    \item \textbf{Участь у форумі:}
    \begin{itemize}
        \item Створення нових тем для обговорення
        \item Публікація повідомлень в існуючих темах
        \item Додавання коментарів до повідомлень
        \item Використання системи вподобань (лайків)
    \end{itemize}
    
    \item \textbf{Взаємодія з базою знань:}
    \begin{itemize}
        \item Перегляд статей та матеріалів
        \item Пошук та фільтрація контенту за категоріями
    \end{itemize}
    
    \item \textbf{Робота з інтерактивною картою:}
    \begin{itemize}
        \item Додавання маркерів вуликів на карту
        \item Перегляд інформації про власні вулики
        \item Редагування та видалення своїх маркерів
        \item Створення полігонів полів з атрибутивною інформацією
        \item Налаштування параметрів полів (тип культури, дати обробки тощо)
        \item Перегляд прогнозу погоди для вибраних локацій
    \end{itemize}
    
    \item \textbf{Використання FAQ-асистента:}
    \begin{itemize}
        \item Надсилання запитань асистенту
        \item Отримання відповідей та рекомендацій
    \end{itemize}
\end{itemize}

\begin{figure}[htbp]
  \centering
  \includegraphics[width=0.85\textwidth]{practice_report/images/usecase_user.png}
  \caption{Діаграма варіантів використання для зареєстрованого користувача}
  \label{fig:usecase_user}
\end{figure}

\subsection{Адміністратор}
Актор з розширеними правами для управління платформою:
\begin{itemize}
    \item Усі можливості зареєстрованого користувача
    \item Доступ до адміністративної панелі
    \item Перегляд списку всіх користувачів
    \item Надання або скасування прав адміністратора іншим користувачам
    \item Модерація контенту (за необхідності)
\end{itemize}

\begin{figure}[htbp]
  \centering
  \includegraphics[width=0.7\textwidth]{practice_report/images/usecase_admin.png}
  \caption{Діаграма варіантів використання для адміністратора}
  \label{fig:usecase_admin}
\end{figure}

Діаграми варіантів використання для кожного типу акторів візуально представляють ці взаємодії, демонструючи повний спектр функціональних можливостей системи з точки зору різних користувачів. Такий підхід дозволяє чітко окреслити межі та можливості платформи на етапі проектування.

\section{Архітектура системи}
\label{sec:architecture}
Розроблений веб-застосунок має класичну клієнт-серверну архітектуру. 

\subsection{Загальна архітектура}
Система складається з трьох основних компонентів: 
\begin{itemize}
    \item Клієнтська частина (Frontend): односторінковий застосунок (Single Page Application, SPA), розроблений на React, відповідає за користувацький інтерфейс та взаємодію з користувачем.
    \item Серверна частина (Backend): RESTful прикладний програмний інтерфейс (Application Programming Interface, API), розроблене на NestJS, відповідає за бізнес-логіку, обробку запитів, взаємодію з базою даних та автентифікацію.
    \item База даних: MongoDB, документо-орієнтована NoSQL база даних, використовується для зберігання всієї інформації застосунку, включаючи дані користувачів, пости форуму, статті бази знань, а також геопросторові дані для карти.
\end{itemize}
Взаємодія між клієнтом та сервером відбувається за протоколом HTTP(S) через RESTful API. Для розгортання використовується Docker, що забезпечує ізоляцію та портативність середовища.

\begin{figure}[htbp]
  \centering
  \includegraphics[width=0.8\textwidth]{practice_report/images/system_architecture.png}
  \caption{Загальна архітектура системи}
  \label{fig:system_architecture}
\end{figure}

\subsection{Архітектура фронтенду}
Клієнтська частина, розроблена на React \cite{react} з використанням TypeScript, становить основу користувацького досвіду платформи. TypeScript забезпечує статичну типізацію, що суттєво підвищує надійність коду та спрощує рефакторинг у великому проекті, особливо при роботі з типами пропсів компонентів, станів та структур даних API. Компонентний підхід React дозволив створити модульну та легко підтримувану структуру UI, де складні інтерфейси, такі як інтерактивна карта або діалогові вікна, розбиваються на менші, незалежні та повторно використовувані компоненти (наприклад, \texttt{AddHiveDialog}, \texttt{EditFieldDialog}).

Управління станом, особливо станом, що надходить з сервера, реалізовано за допомогою Redux Toolkit \cite{reduxtoolkit}, зокрема його інструменту RTK Query. Цей вибір дозволив значно спростити логіку взаємодії з REST API бекенду, автоматизувавши процеси запиту даних, їх кешування, оновлення при змінах (invalidatesTags) та обробку станів завантаження та помилок. Для таких сутностей, як вулики, поля та профілі користувачів, RTK Query автоматично генерує хуки (наприклад, \texttt{useGetHivesQuery}, \texttt{useAddHiveMutation}), що мінімізує кількість шаблонного коду.

Для побудови користувацького інтерфейсу було обрано бібліотеку компонентів Material-UI (MUI) \cite{materialui}. Широкий набір готових, кастомізованих та адаптивних компонентів (кнопки, форми, діалоги, сітки, іконки) прискорив розробку та забезпечив консистентний візуальний стиль, що відповідає принципам Material Design. Можливості темизації MUI також були використані для адаптації колірної схеми до тематики платформи.

Навігація в рамках односторінкового застосунку (SPA) реалізована за допомогою бібліотеки React Router. Вона дозволяє визначати маршрути для різних сторінок (наприклад, \texttt{/map}, \texttt{/profile}, \texttt{/forums}) та управляти переходами між ними без перезавантаження сторінки, що є стандартом для сучасних веб-застосунків.

Картографічний функціонал, що є центральним для платформи, побудований на базі бібліотеки Leaflet \cite{leaflet} та її React-обгортки React-Leaflet. Це поєднання надає декларативний спосіб інтеграції інтерактивних карт в React-компоненти, дозволяючи легко управляти шарами, маркерами (як стандартними, так і кастомними), полігонами та спливаючими вікнами.

Інтернаціоналізація інтерфейсу для підтримки української та англійської мов реалізована за допомогою бібліотеки i18next \cite{i18next} та її інтеграції з React (react-i18next), що дозволяє зберігати текстові ресурси в окремих файлах та динамічно змінювати мову застосунку.

Загальна структура коду фронтенду організована за функціональними та типовими ознаками, включаючи директорії для компонентів (загальних та специфічних для модулів, наприклад \texttt{map/}), сторінок, API-сервісів (зрізів RTK Query у \texttt{store/api/}), кастомних хуків, контекстів (наприклад, \texttt{AuthContext}) та утиліт.

\subsection{Архітектура бекенду}
Серверна частина розроблена на платформі Node.js з використанням фреймворку NestJS \cite{nestjs} та TypeScript. Ключовою перевагою NestJS для даного проекту є його модульна архітектура, що забезпечує чітке розділення відповідальностей та сприяє високій підтримуваності коду. Кожен основний функціональний блок платформи, такий як автентифікація (\texttt{AuthModule}), управління користувачами (\texttt{UsersModule}), форумом (\texttt{ForumModule}), картографічними об'єктами (\texttt{HivesModule}, \texttt{FieldsModule}), та нещодавно доданий FAQ-асистент (\texttt{FaqModule}), реалізований як окремий модуль NestJS.

Кожен такий модуль інкапсулює власні компоненти:
\begin{itemize}
    \item \textbf{Контролери (Controllers):} Обробляють вхідні HTTP-запити, валідують їх (часто за допомогою DTO та автоматичних пайпів валідації NestJS), викликають відповідні методи сервісів та повертають відповіді клієнту. Наприклад, \texttt{HivesController} містить ендпоінти для створення, отримання, оновлення та видалення вуликів.
    \item \textbf{Сервіси (Services):} Містять основну бізнес-логіку модуля. Вони взаємодіють з базою даних через моделі Mongoose, виконують операції з даними, реалізують специфічні для домену правила та логіку. Наприклад, \texttt{AuthService} відповідає за валідацію користувачів, генерацію JWT токенів та взаємодію з \texttt{UsersService} для створення нових користувачів або перевірки їх статусу.
    \item \textbf{Об'єкти Передачі Даних (Data Transfer Objects - DTOs):} Використовуються для визначення структури даних, що передаються між клієнтом та сервером (в тілах запитів) або між різними шарами застосунку. DTO, визначені за допомогою класів TypeScript та декораторів з бібліотеки \texttt{class-validator}, дозволяють автоматично валідувати вхідні дані на рівні контролера за допомогою вбудованого в NestJS \texttt{ValidationPipe}. Це забезпечує, що до сервісів потрапляють лише коректно сформовані дані, підвищуючи надійність API. Прикладом є \texttt{CreateHiveDto}, що валідує поля, необхідні для створення нового вулика.
    \item \textbf{Схеми Mongoose (Schemas):} Визначають структуру документів у відповідних колекціях MongoDB та правила їх валідації на рівні бази даних. Наприклад, \texttt{HiveSchema} визначає поля для назви, нотаток, геокоординат та посилання на користувача.
    \item \textbf{Guards, Strategies, Decorators (в основному в AuthModule):} NestJS надає потужні механізми для реалізації автентифікації та авторизації. В \texttt{AuthModule} використовуються Passport.js стратегії (\texttt{LocalStrategy}, \texttt{JwtStrategy}, \texttt{GoogleStrategy}), guards (\texttt{JwtAuthGuard}, \texttt{AdminGuard}) для захисту маршрутів, та кастомні декоратори (наприклад, \texttt{@GetUser()}) для зручного доступу до даних користувача в контролерах.
\end{itemize}
Для взаємодії з базою даних MongoDB \cite{mongodb} використовується ODM Mongoose \cite{mongoose}, що дозволяє працювати з даними в об'єктно-орієнтованому стилі. Автоматична генерація документації API за допомогою Swagger (на базі специфікації OpenAPI \cite{openapi}) значно спрощує тестування та інтеграцію з клієнтською частиною. Використання Fastify \cite{fastify} як HTTP-адаптера для NestJS було обрано з метою підвищення продуктивності обробки запитів порівняно зі стандартним Express адаптером.

\subsection{Схема бази даних}
\label{subsec:db_schema_detailed}
Для зберігання даних застосунку – веб-платформи для комунікації та обміну знаннями в спільноті бджолярів \textit{Beekeepers Community Platform} – використовується документо-орієнтована NoSQL база даних MongoDB \cite{mongodb}, а взаємодія з нею на рівні NestJS-сервісів реалізована за допомогою об'єктно-документного відображення (Object Document Mapper, ODM) Mongoose \cite{mongoose}. Такий підхід забезпечує гнучкість у структуруванні даних та зручні інструменти для їх валідації та маніпуляції. Нижче описано структуру основних колекцій бази даних.

\subsubsection*{Колекція \texttt{users}}
Зберігає інформацію про зареєстрованих користувачів платформи. Кожен документ у колекції має наступні ключові поля:
\begin{itemize}
    \item \texttt{email} (String): Електронна пошта користувача, використовується як логін. Поле є обов'язковим та унікальним.
    \item \texttt{password} (String): Хеш паролю користувача. Поле є обов'язковим.
    \item \texttt{username} (String): Ім'я користувача, що відображається на платформі. Поле є обов'язковим.
    \item \texttt{bio} (String, optional): Коротка біографія або опис користувача.
    \item \texttt{location} (String, optional): Місцезнаходження користувача.
    \item \texttt{expertise} (Array of Strings, optional): Список сфер експертизи бджоляра.
    \item \texttt{isEmailVerified} (Boolean): Прапорець, що вказує, чи підтвердив користувач свою електронну пошту. За замовчуванням \texttt{false}.
    \item \texttt{emailVerificationToken} (String, optional): Токен для верифікації email. Не вибирається за замовчуванням при запитах.
    \item \texttt{emailVerificationExpires} (Date, optional): Термін дії токена верифікації. Не вибирається за замовчуванням.
    \item \texttt{isAdmin} (Boolean): Прапорець адміністратора. За замовчуванням \texttt{false}.
    \item \texttt{createdAt}, \texttt{updatedAt} (Date): Часові мітки, що автоматично додаються Mongoose завдяки опції \texttt{{timestamps: true}}.
\end{itemize}

\subsubsection*{Колекція \texttt{hives}}
Призначена для зберігання інформації про вулики, додані користувачами на інтерактивну карту.
\begin{itemize}
    \item \texttt{name} (String): Назва вулика. Поле є обов'язковим.
    \item \texttt{notes} (String, optional): Додаткові нотатки або опис вулика.
    \item \texttt{location} (Object): Геопросторові дані про місцезнаходження вулика. Вбудований об'єкт типу \texttt{Point} (GeoJSON), що містить:
    \begin{itemize}
        \item \texttt{type} (String): Тип геометрії, фіксоване значення \texttt{'Point'}. Обов'язкове.
        \item \texttt{coordinates} (Array of Numbers): Масив з двох чисел [довгота, широта]. Обов'язкове.
    \end{itemize}
    Поле \texttt{location} індексується за допомогою \texttt{2dsphere} індексу для ефективних геопросторових запитів.
    \item \texttt{user} (ObjectId): Ідентифікатор користувача-власника вулика, посилається на колекцію \texttt{users}. Поле є обов'язковим та індексованим.
    \item \texttt{createdAt}, \texttt{updatedAt} (Date): Автоматичні часові мітки.
\end{itemize}

\subsubsection*{Колекція \texttt{fields}}
Зберігає інформацію про сільськогосподарські поля, які користувачі позначають на карті.
\begin{itemize}
    \item \texttt{name} (String): Назва поля. Поле є обов'язковим.
    \item \texttt{cropType} (String): Тип культури, що вирощується на полі. Поле є обов'язковим.
    \item \texttt{bloomingPeriodStart} (Date): Дата початку періоду цвітіння культури. Поле є обов'язковим.
    \item \texttt{bloomingPeriodEnd} (Date): Дата кінця періоду цвітіння культури. Поле є обов'язковим.
    \item \texttt{treatmentDates} (Array of Dates, optional): Список запланованих дат обробки поля. За замовчуванням порожній масив.
    \item \texttt{geometry} (Object): Геометрія поля. Вбудований об'єкт типу \texttt{Polygon} (GeoJSON), що містить:
    \begin{itemize}
        \item \texttt{type} (String): Тип геометрії, фіксоване значення \texttt{'Polygon'}. Обов'язкове.
        \item \texttt{coordinates} (Array of Array of Array of Numbers): Масив координат, що визначають полігон (стандарт GeoJSON [[[lng, lat], ...]]). Обов'язкове.
    \end{itemize}
    Поле \texttt{geometry} індексується за допомогою \texttt{2dsphere} індексу.
    \item \texttt{user} (ObjectId): Ідентифікатор користувача, який додав поле. Посилається на колекцію \texttt{users}. Поле є обов'язковим та індексованим.
    \item \texttt{createdAt}, \texttt{updatedAt} (Date): Автоматичні часові мітки.
\end{itemize}

\subsubsection*{Колекція \texttt{forumposts}}
Містить пости, створені користувачами на форумі платформи.
\begin{itemize}
    \item \texttt{title} (String): Заголовок поста. Поле є обов'язковим.
    \item \texttt{content} (String): Основний зміст поста. Поле є обов'язковим.
    \item \texttt{author} (ObjectId): Ідентифікатор автора поста, посилається на колекцію \texttt{users}. Поле є обов'язковим.
    \item \texttt{likes} (Array of ObjectId, optional): Масив ідентифікаторів користувачів, які вподобали пост. Посилається на колекцію \texttt{users}. За замовчуванням порожній масив.
    \item \texttt{comments} (Array of Objects, optional): Масив коментарів до поста. Кожен об'єкт коментаря містить:
    \begin{itemize}
        \item \texttt{content} (String): Текст коментаря. Обов'язкове.
        \item \texttt{author} (ObjectId): Ідентифікатор автора коментаря, посилається на колекцію \texttt{users}. Обов'язкове.
        \item \texttt{createdAt} (Date): Дата створення коментаря. За замовчуванням поточна дата.
    \end{itemize}
    \item \texttt{createdAt}, \texttt{updatedAt} (Date): Часові мітки для самого поста, автоматично керовані Mongoose (також можуть бути явно визначені в схемі, як вказано).
\end{itemize}
Така деталізована схема даних забезпечує зберігання всієї необхідної інформації для функціонування платформи та підтримує специфічні вимоги, такі як геопросторові запити та зв'язки між різними сутностями.

\begin{figure}[htbp]
  \centering
  \includegraphics[width=0.9\textwidth]{practice_report/images/db_schema.png}
  \caption{Логічна схема бази даних}
  \label{fig:db_schema}
\end{figure}

\section{Проектування UI/UX}
\label{sec:ui_ux}
Проектування користувацького інтерфейсу (User Interface, UI) та досвіду взаємодії (User Experience, UX) було спрямоване на створення інтуїтивно зрозумілої, зручної та візуально привабливої платформи. Основним інструментом для реалізації UI стала бібліотека компонентів Material-UI, яка надає широкий набір готових елементів дизайну, що відповідають сучасним стандартам Material Design. 

Ключові рішення:
\begin{itemize}
    \item \textbf{Адаптивний дизайн:} Забезпечення коректного відображення та функціонування на різних пристроях (десктопи, планшети, мобільні телефони).
    \item \textbf{Інтуїтивна навігація:} Використання бічної панелі навігації для доступу до основних розділів сайту та чіткої ієрархії сторінок.
    \item \textbf{Консистентність інтерфейсу:} Дотримання єдиного стилю оформлення елементів на всіх сторінках застосунку.
    \item \textbf{Інтерактивність:} Надання користувачам можливості легко взаємодіяти з елементами, такими як карта, форми, кнопки.
    \item \textbf{Зворотний зв'язок:} Інформування користувача про результати його дій (успіх, помилка, процес завантаження) за допомогою повідомлень та індикаторів.
\end{itemize}
% TODO: Include wireframes or mockups if available (can be in appendix). 

\subsubsection{Вимоги до форуму}
\begin{itemize}
    \item Реєстрація користувачів з верифікацією електронної пошти.
    \item Автентифікація (логін/пароль, можливість входу через Google OAuth).
    \item Зберігання даних користувачів (email, хеш паролю, ім'я користувача, роль, інформація про пасіку за бажанням).
    \item Управління профілем користувача.
    \item Відновлення паролю.
\end{itemize}

\subsubsection{Вимоги до бази знань}
\begin{itemize}
    \item Створення, редагування та видалення статей/ресурсів адміністраторами.
    \item Категоризація та тегування матеріалів.
    \item Пошук по базі знань.
    \item Система коментарів до статей (за бажанням).
    \item Рейтинг статей (за бажанням).
\end{itemize}

\subsubsection{Загальні нефункціональні вимоги}
\begin{itemize}
    \item \textbf{Безпека:} Захист від основних веб-вразливостей (Cross-Site Scripting (XSS), Cross-Site Request Forgery (CSRF), SQL/NoSQL ін\'єкції), безпечне зберігання паролів, використання HTTPS.
    \item \textbf{Продуктивність:} Швидке завантаження сторінок та відгук інтерфейсу.
    \item \textbf{Масштабованість:} Архітектура повинна дозволяти додавання нового функціоналу та витримувати зростання кількості користувачів.
    \item \textbf{Надійність:} Система повинна бути доступною та стабільно працювати.
    \item \textbf{Зручність використання (Usability):} Інтуїтивно зрозумілий та легкий у використанні інтерфейс.
    \item \textbf{Адаптивність (Responsiveness):} Коректне відображення на різних пристроях (десктопи, планшети, мобільні телефони).
    \item \textbf{Інтернаціоналізація (i18n):} Підтримка української та англійської мов інтерфейсу.
    \item \textbf{Зворотний зв'язок:} Інформування користувача про результати його дій, помилки.
\end{itemize} 