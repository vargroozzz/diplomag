% --- Appendices ---
\appendix % Starts appendix numbering (A, B, C...)
\addcontentsline{toc}{chapter}{\appendixname}
\chapter{Приклад коду API}
\label{app:api_code_sample}

% TODO: Include a relevant snippet of backend API code (e.g., a controller method)
\begin{verbatim}
// Example: Backend - HivesController - create method
@Post()
@Version('1')
@ApiOperation({ summary: 'Create a new hive' })
@ApiResponse({ status: 201, description: 'Hive created successfully.' })
@ApiResponse({ status: 400, description: 'Invalid input.' })
create(@Body() createHiveDto: CreateHiveDto, @GetUser() user: JwtUserPayload) {
  return this.hivesService.create(createHiveDto, user.userId);
}
\end{verbatim}

\chapter{Документація API (Swagger)}
\label{app:swagger_docs}
% TODO: Include a screenshot of Swagger UI or link if publicly accessible
Документація API була автоматично згенерована за допомогою Swagger (OpenAPI) і доступна за ендпоінтом /docs на сервері розробки.
\begin{figure}[H]
  \centering
  % \includegraphics[width=0.9\textwidth]{images/swagger_screenshot.png} % Replace with your screenshot
  \fbox{Placeholder for Swagger UI Screenshot}
  \caption{Приклад інтерфейсу Swagger UI для API}
  \label{fig:swagger}
\end{figure}

\chapter{Інструкція користувача}
\label{app:user_manual}
% TODO: Provide a brief user manual for key features.

\section*{Реєстрація та вхід}
Опис процесу реєстрації та входу, включаючи верифікацію email та вхід через Google.

\section*{Використання форуму}
Як створювати теми, писати повідомлення, коментувати.

\section*{Робота з картою}
Як додавати та переглядати вулики та поля. 