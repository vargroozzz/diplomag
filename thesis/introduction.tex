% --- Introduction ---
\chapter*{Вступ}
\addcontentsline{toc}{chapter}{Вступ}
\label{ch:introduction}

Актуальність теми магістерської роботи зумовлена зростаючою потребою у спеціалізованих онлайн-платформах для нішевих спільнот, зокрема для бджолярів. Бджільництво є важливою галуззю сільського господарства та екології, і ефективний обмін знаннями, досвідом та оперативною інформацією між пасічниками може суттєво сприяти його розвитку. Існуючі загальні соціальні мережі та форуми не завжди враховують специфічні потреби бджолярської спільноти, такі як обговорення хвороб бджіл, методів догляду, медоносних рослин, а також координація дій щодо обробки полів та розташування пасік.

Метою даної магістерської роботи є розробка фулстек веб-застосунку \textit{Beekeepers Community Platform}, що надасть бджолярам зручні інструменти для комунікації, обміну інформацією, доступу до бази знань та управління даними про власні пасіки та сільськогосподарські угіддя.

Для досягнення поставленої мети було визначено наступні завдання:
\begin{itemize}
    \item Провести аналіз предметної області та існуючих рішень.
    \item Обґрунтувати вибір технологічного стеку для розробки.
    \item Спроектувати архітектуру клієнтської та серверної частин застосунку, а також схему бази даних.
    \item Реалізувати основний функціонал платформи, включаючи систему реєстрації та автентифікації, форум, базу знань та інтерактивну карту.
    \item Забезпечити базові механізми безпеки та валідації даних.
    \item Розробити інтуїтивно зрозумілий та адаптивний користувацький інтерфейс.
\end{itemize}

Об'єктом дослідження є процес проектування та розробки повностекового веб-застосунку для нішевої спільноти.

Предметом дослідження є архітектурні рішення, технології та інструменти для створення інтерактивної та функціональної платформи для бджолярів, що включає засоби комунікації, обміну знаннями та геоінформаційні функції.

Методи дослідження, що використовувалися в роботі, включають:
\begin{itemize}
    \item Аналіз науково-технічної літератури та існуючих аналогів.
    \item Системний аналіз та проектування програмного забезпечення.
    \item Об'єктно-орієнтоване програмування.
    \item Використання сучасних веб-технологій та фреймворків: React для розробки клієнтської частини, NestJS (Node.js) для серверної частини, MongoDB як система управління базами даних.
    \item Застосування бібліотек Material-UI для користувацького інтерфейсу та Leaflet для реалізації картографічного функціоналу.
\end{itemize}

Наукова новизна (якщо є, сформулювати).
Практичне значення отриманих результатів полягає у створенні готового до використання прототипу веб-платформи, що може бути впроваджена для підтримки спільноти бджолярів, покращення їх взаємодії та доступу до актуальної інформації.

Апробація результатів роботи (якщо є, вказати конференції, публікації).

Структура роботи: Магістерська робота складається зі вступу, чотирьох розділів, висновків, списку використаних джерел та додатків. 