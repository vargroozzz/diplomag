\section*{Тези доповіді на наукову конференцію}
\addcontentsline{toc}{section}{Тези доповіді на наукову конференцію}
\label{sec:conference_abstract}

\subsection*{Вступ}
Бджільництво, важлива складова аграрного сектору та ключовий елемент підтримки екологічного балансу, стикається з низкою сучасних викликів, включаючи поширення хвороб бджіл, вплив пестицидів та необхідність швидкого обміну достовірною інформацією. Ефективна комунікація та координація дій між пасічниками часто ускладнені через відсутність спеціалізованих цифрових інструментів. Дана робота представляє веб-платформу для комунікації та обміну знаннями в спільноті бджолярів \textit{Beekeepers Community Platform} – повностековий веб-застосунок, розроблений для задоволення потреб української спільноти бджолярів. Платформа надає єдине інтегроване середовище для обміну інформацією, управління геоданими пасік та полів, а також доступу до профільної бази знань, сприяючи вирішенню зазначених проблем.

\subsection*{Методологія та технологічний стек}
Розробка платформи базувалася на ітеративному підході, що дозволяв поступово реалізовувати та тестувати функціональні модулі. Було обрано сучасний стек технологій, орієнтований на продуктивність, масштабованість та зручність розробки. 
\textbf{Клієнтська частина (Frontend)} реалізована на React \cite{react} з використанням TypeScript, Vite \cite{vite} для збірки, Material-UI \cite{materialui} для користувацького інтерфейсу, Redux Toolkit (RTK Query) \cite{reduxtoolkit} для управління станом та взаємодії з API, і Leaflet \cite{leaflet} (з React-Leaflet) для картографічного функціоналу. 
\textbf{Серверна частина (Backend)} побудована на Node.js з фреймворком NestJS \cite{nestjs} (з адаптером Fastify \cite{fastify}) та TypeScript, що забезпечує модульну та типізовану архітектуру. Для автентифікації використовується Passport.js \cite{passportjs}.
\textbf{База даних} – MongoDB \cite{mongodb} (з Mongoose ODM \cite{mongoose}), обрана через гнучкість схеми та ефективну роботу з GeoJSON даними.
API розроблено за принципами RESTful з версіонуванням та документується за допомогою OpenAPI \cite{openapi}. Проєкт також включає інтернаціоналізацію (i18next \cite{i18next}) та підготовлений до контейнеризації (Docker \cite{docker}) і розгортання на PaaS-платформі Render \cite{renderpaas}.

\subsection*{Реалізований функціонал}
Ключовими реалізованими модулями платформи, що формують її основну цінність для користувачів, є:
\begin{itemize}
    \item \textbf{Система автентифікації та управління користувачами:} Забезпечує безпечний доступ до платформи через реєстрацію з верифікацією електронної пошти (включаючи повторне надсилання листа верифікації та обмеження дії токена), вхід з використанням JWT (access та refresh токени), та можливість швидкої автентифікації через Google OAuth. Користувачі мають можливість переглядати та оновлювати свої профілі, що включають специфічну для бджолярів інформацію.
    \item \textbf{Форум для обговорень:} Інтерактивний модуль для створення тематичних обговорень, публікації інформативних повідомлень, залишення коментарів та висловлення реакцій (система вподобань), що сприяє активній взаємодії та обміну досвідом у спільноті. (Поточна реалізація включає основні CRUD операції для постів та коментарів).
    \item \textbf{Інтерактивна карта (Управління пасіками та полями):} Центральний геоінформаційний компонент платформи, що дозволяє користувачам ефективно управляти просторовими даними:
        \begin{itemize}
            \item Додавання, перегляд та видалення (з діалоговим підтвердженням) маркерів вуликів, з використанням кастомних іконок (MUI \texttt{HiveIcon}) для покращеної візуальної ідентифікації.
            \item Додавання та редагування метаданих полігональних об'єктів полів, включаючи назву, тип вирощуваної культури, період цвітіння та список запланованих дат обробки.
            \item Динамічна візуалізація статусу обробки полів за допомогою інтуїтивного кольорового кодування полігонів (наприклад, червоний – обробка сьогодні, помаранчевий – обробка протягом найближчих 7 днів), що слугує системою раннього попередження.
            \item Відображення детальної інформації про обрані об'єкти у спливаючих вікнах (Popups) безпосередньо на карті.
        \end{itemize}
    \item \textbf{База знань та FAQ-асистент:} Реалізовано прототип модуля бази знань для доступу до статей та ресурсів. Додатково розроблено інтелектуального FAQ-асистента на базі OpenAI GPT-3.5-turbo, що надає відповіді на питання користувачів, спираючись на наданий контекст ЧаПи, демонструючи потенціал ШІ для підтримки користувачів.
\end{itemize}

\subsection*{Ключові технічні рішення та аспекти реалізації}
В процесі розробки було застосовано та реалізовано низку ефективних технічних рішень, що забезпечили функціональність та якість платформи:
\begin{itemize}
    \item Інтеграція React-компонентів (MUI \texttt{HiveIcon}) в картографічну бібліотеку Leaflet за допомогою 
\texttt{ReactDOMServer.renderToString} та \texttt{L.divIcon}, що дозволило створити кастомізовані та тематично відповідні маркери вуликів.
    \item Розробка клієнтської логіки в компоненті \texttt{MapPage.tsx} для динамічного розрахунку та зміни кольору полігонів полів на основі дат їх обробки, що підвищило інформативність та практичну цінність карти для бджолярів.
    \item Активне використання RTK Query для оптимізації управління станом API, включаючи автоматичне кешування даних, підтримку оптимістичних оновлень для CRUD операцій з вуликами та полями, та ефективну обробку станів завантаження і помилок.
    \item Забезпечення надійної доставки транзакційних електронних листів (для верифікації акаунтів та інших сповіщень) через інтеграцію з сервісом Mailgun та коректним налаштуванням відповідних DNS записів (SPF, DKIM, DMARC) для підвищення доставляємості та запобігання спаму.
    \item Впровадження автоматизованих перевірок якості коду за допомогою GitHub Actions для лінтингу на Pull Requests, що сприяє підтримці консистентності та чистоти кодової бази.
\end{itemize}

\subsection*{Результати та обговорення}
Розроблений веб-застосунок – веб-платформа для комунікації та обміну знаннями в спільноті бджолярів \textit{Beekeepers Community Platform} – є функціональним прототипом, що успішно реалізує заявлені ключові можливості та демонструє потенціал для створення цінного ресурсу для спільноти. Платформа надає зручний та інтуїтивно зрозумілий інтерфейс, а також необхідні інструменти для бджолярів, що сприяють ефективному обміну досвідом, покращенню координації та оперативному інформуванню. Особливу практичну цінність становить інтерактивна карта з можливістю управління геопросторовими даними пасік та полів, а також інноваційна візуалізація потенційно небезпечних періодів обробки полів, що напряму впливає на безпеку бджільництва.

\subsection*{Основні наукові та інноваційні результати роботи}
У ході виконання науково-дослідницької практики та розробки платформи було отримано наступні ключові інноваційні результати:
\begin{itemize}
    \item Створено архітектуру та реалізовано прототип унікального для України комплексного веб-рішення – веб-платформи для комунікації та обміну знаннями в спільноті бджолярів \textit{Beekeepers Community Platform}, яке інтегрує форум для спілкування, спеціалізовану базу знань та інструменти геопросторового менеджменту (інтерактивна карта пасік/полів), спеціально адаптовані до потреб вітчизняних бджолярів.
    \item Підвищено оперативну обізнаність бджолярів шляхом розробки та впровадження оригінального методу динамічного кольорового кодування полігонів на інтерактивній карті, що візуалізує ризики, пов\'язані із запланованими обробками сільськогосподарських полів, та слугує інструментом їх попередження.
    \item Здійснено успішне практичне застосування сучасного технологічного стеку (React, NestJS, Leaflet, MongoDB) для побудови функціональної та масштабованої агро-геоінформаційної платформи для нішевої спільноти, що підтверджує його придатність для вирішення подібних завдань в контексті цифрової трансформації сільського господарства та управління онлайн-спільнотами \cite{preece2005onlinecommunities, huet2022digitalbeekeeping, guruprasad2024beeopen}.
    \item Інтегровано прототип інноваційного FAQ-асистента на базі великої мовної моделі (LLM) з кастомізованим промпт-інжинірингом, що забезпечує контекстно-залежну інформаційну підтримку користувачів у специфічній сфері бджільництва.
\end{itemize}

\subsection*{Висновки та напрямки подальшого розвитку}
Створена в рамках науково-дослідницької практики веб-платформа для комунікації та обміну знаннями в спільноті бджолярів \textit{Beekeepers Community Platform} успішно демонструє життєздатність запропонованої концепції та вирішує поставлені завдання щодо надання спеціалізованого цифрового інструменту для бджолярів. Робота дозволила отримати практичний досвід у всіх етапах розробки повностекового веб-застосунку. Платформа має значний потенціал для подальшого розвитку. Першочерговими напрямками є розширення функціоналу форуму та бази знань, впровадження системи проактивних сповіщень (зокрема, про обробку полів поблизу пасік), розробка мобільного застосунку для підвищення доступності, а також інтеграція з іншими сервісами, корисними для бджолярів (наприклад, погодні сервіси, аналітика медозбору, ринкові дані).

\textbf{Ключові слова:} веб-застосунок, спільнота бджолярів, React, NestJS, MongoDB, форум, інтерактивна карта, GeoJSON, автентифікація, Leaflet, RTK Query, управління пасіками, OpenAI.

\newpage 