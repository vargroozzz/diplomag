% --- Chapter 4: Testing and Deployment ---
\chapter{Тестування та розгортання}
\label{ch:testing_deployment}

\section{Тестування системи}
\label{sec:testing}
Тестування є невід'ємною частиною процесу розробки програмного забезпечення, спрямованою на виявлення помилок та забезпечення відповідності функціональним та нефункціональним вимогам. Для веб-застосунку \textit{Beekeepers Community Platform} було проведено кілька видів тестування.

\subsection{Модульне тестування (Unit Testing)}
Модульне тестування було зосереджено на перевірці окремих компонентів та функцій серверної частини (NestJS). Використовувався вбудований в NestJS тестовий фреймворк, що базується на Jest. Тестувалися сервіси, контролери (частково) та допоміжні утиліти. Основна увага приділялася тестуванню бізнес-логіки в сервісах, валідації даних та коректності відповідей API.
% TODO: Provide examples of unit tests if relevant or link to appendix.

\subsection{Інтеграційне тестування}
Інтеграційне тестування передбачало перевірку взаємодії між різними модулями системи, зокрема між фронтендом та бекендом (API ендпоінти), а також взаємодію бекенду з базою даних MongoDB. На цьому етапі перевірялася коректність обробки запитів, передачі даних та їх збереження/отримання.

\subsection{Тестування користувацького інтерфейсу (UI Testing)}
На фронтенді проводилося ручне тестування користувацького інтерфейсу на різних пристроях та в різних браузерах (Chrome, Firefox, Safari) для забезпечення адаптивності та коректного відображення. Перевірялася робота інтерактивних елементів, форм, навігації та картографічного функціоналу.
% TODO: Mention tools if any were used for automated UI testing (e.g., Cypress, Playwright - even if just planned).

\subsection{Тестування безпеки}
Здійснювалася базова перевірка на поширені веб-вразливості, такі як XSS (здебільшого покривається React), валідація вхідних даних для запобігання ін'єкціям на рівні API. Також перевірялася робота системи автентифікації та авторизації, зокрема захист маршрутів та валідність JWT токенів.

\section{Розгортання застосунку}
\label{sec:deployment}
Для розгортання веб-застосунку \textit{Beekeepers Community Platform} використовується контейнеризація за допомогою Docker та Docker Compose. Це дозволяє створити ізольоване та відтворюване середовище для роботи застосунку, що спрощує процес розгортання та управління залежностями.

Створено \texttt{Dockerfile} для клієнтської частини (React/Vite) та для серверної частини (NestJS). Файл \texttt{docker-compose.yml} описує сервіси для фронтенду, бекенду та бази даних MongoDB, а також налаштовує їх взаємодію та мережеві налаштування.

Для продакшн-розгортання рекомендується використання оберненого проксі-сервера (наприклад, Nginx) для обслуговування статичних файлів фронтенду, кешування, балансування навантаження (за потреби) та налаштування HTTPS.
% TODO: Detail build process, CI/CD pipeline if set up (even basic).
% TODO: Mention hosting provider considerations (e.g., VPS, PaaS like Heroku, Render, Fly.io).

\section{Майбутні напрямки розвитку}
\label{sec:future_work}
Платформа \textit{Beekeepers Community Platform} має потенціал для подальшого розвитку та розширення функціоналу. Можливі напрямки включають:
\begin{itemize}
    \item Розширення функціоналу бази знань: додавання можливості користувачам пропонувати статті, система рецензування, коментарі.
    \item Розвиток картографічного сервісу: фільтри за типами культур, періодами цвітіння, сповіщення про обробку полів, інтеграція з погодними даними.
    \item Система приватних повідомлень між користувачами.
    \item Календар подій для бджолярів (виставки, ярмарки, семінари).
    \item Мобільний застосунок (React Native або нативні технології).
    \item Розширена аналітика та статистика для користувачів (наприклад, продуктивність пасік).
    \item Інтеграція з іншими сервісами (наприклад, маркетплейси для продукції бджільництва).
\end{itemize} 