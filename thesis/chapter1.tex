% --- Chapter 1: Analysis of the Subject Area ---
\chapter{Аналіз предметної області}
\label{ch:analysis}

\section{Огляд існуючих рішень}
\label{sec:existing_solutions}
% TODO: Review existing forums, social platforms for niche communities, beekeeping specific tools.
% Compare features, pros, and cons.
Даний розділ присвячено аналізу існуючих веб-платформ та мобільних застосунків, що можуть використовуватися спільнотами для обміну інформацією, а також спеціалізованих рішень для бджолярів та ширшого агросектору. Розглядаються популярні форумні системи, соціальні мережі для груп за інтересами, а також застосунки, що пропонують інструменти для ведення пасіки чи моніторингу умов медозбору. Важливість розробки таких систем підкреслюється дослідженнями в галузі агро-геоінформатики, що фокусуються на зборі, обробці та візуалізації геопросторових даних для підтримки прийняття рішень в сільському господарстві \cite{granell2015agrogeoinformatics}. Прикладами спеціалізованих інформаційно-комунікаційних технологій (ІКТ)-проектів для бджільництва є ініціативи зі створення відкритих екосистем обміну даними \cite{guruprasad2024beeopen}, дослідження цифрової трансформації галузі через архітектури підтримки прийняття рішень \cite{huet2022digitalbeekeeping}, а також проекти, спрямовані на розробку сервісів для розумного управління пасіками, зокрема в країнах, що розвиваються \cite{wakjira2021sams}. Дослідження в галузі онлайн-спільнот підкреслюють важливість дизайну, теорії та практики для їх успішного функціонування та розвитку \cite{preece2005onlinecommunities}. Аналіз включає порівняння їх функціональних можливостей, переваг та недоліків у контексті потреб української спільноти бджолярів.

\section{Порівняння технологій для веб-розробки}
\label{sec:tech_comparison}
% TODO: Discuss frontend (React vs Angular/Vue), backend (NestJS vs Express/Django/Spring), database (MongoDB vs PostgreSQL/MySQL).
% Focus on aspects relevant to this project: community features, real-time (optional), map integration, scalability.
Вибір технологічного стеку є ключовим етапом розробки будь-якого програмного продукту. Для проекту веб-платформи для комунікації та обміну знаннями в спільноті бджолярів \textit{Beekeepers Community Platform} розглядалися наступні категорії технологій:

\subsection{Технології фронтенду}
Для розробки клієнтської частини веб-застосунку основними кандидатами були такі JavaScript-бібліотеки та фреймворки як React, Angular та Vue.js.
\subsubsection{React}
React є популярною JavaScript-бібліотекою для створення користувацьких інтерфейсів, розробленою Facebook \cite{react}. Її перевагами є компонентний підхід, велика екосистема, значна спільнота розробників, гнучкість та висока продуктивність завдяки використанню віртуальної моделі документа (Document Object Model, DOM). Для даного проекту React був обраний через його здатність ефективно будувати складні та інтерактивні інтерфейси, велику кількість готових компонентів (зокрема, інтеграція з Material-UI та картографічними бібліотеками як Leaflet), а також через наявний досвід розробника з цією технологією.

\subsubsection{Angular}
Angular — це комплексний фреймворк від Google, що надає повний набір інструментів для створення великих корпоративних застосунків. Хоча Angular є потужним рішенням, для даного проекту його всеохоплююча структура та дещо вищий поріг входження були розцінені як надлишкові.

\subsubsection{Vue.js}
Vue.js — прогресивний JavaScript-фреймворк, відомий своєю простотою інтеграції та низьким порогом входження. Він є гарним вибором для багатьох проектів, однак екосистема React та доступність спеціалізованих бібліотек для React виявилися більш привабливими для специфічних завдань проекту (наприклад, використання Redux Toolkit Query (RTK Query) для управління станом прикладного програмного інтерфейсу (API)).

\subsection{Технології бекенду}
Для серверної частини розглядалися Node.js-фреймворки NestJS та Express.js, а також Django на Python.
\subsubsection{NestJS (Node.js)}
NestJS — це прогресивний фреймворк для створення ефективних, масштабованих серверних застосунків на Node.js \cite{nestjs}. Він побудований з використанням TypeScript та поєднує елементи об'єктно-орієнтованого програмування, функціонального програмування та функціонально-реактивного програмування. Архітектура NestJS, що базується на модулях, контролерах та сервісах, сприяє чіткій організації коду та його підтримці. Вбудована підтримка TypeScript, легка інтеграція з Passport.js для автентифікації, Swagger для документації API та TypeORM/Mongoose для роботи з базами даних роблять його чудовим вибором для розробки REST API. Обраний для проекту завдяки модульній структурі, підтримці TypeScript та хорошій інтеграції з інструментами, необхідними для проекту.

\subsubsection{Express.js (Node.js)}
Express.js є мінімалістичним та гнучким Node.js веб-фреймворком. Він надає базовий набір функцій для веб-застосунків та API, але вимагає більше ручного налаштування та структурування проекту порівняно з NestJS. Для проекту, що передбачає розширення функціоналу, більш структурований підхід NestJS був визнаний кращим.

\subsubsection{Django (Python)}
Django — високорівневий Python веб-фреймворк, що заохочує швидку розробку та чистий, прагматичний дизайн. Він включає багато готових компонентів, таких як об'єктно-реляційне відображення (Object-Relational Mapping, ORM), адміністративна панель тощо. Однак, для даного проекту перевага була надана стеку на JavaScript/TypeScript для забезпечення однорідності технологій на фронтенді та бекенді.

\subsection{Системи управління базами даних}
Вибір системи управління базами даних (СУБД) ґрунтувався на потребах гнучкості схеми даних та роботи з геопросторовими даними.
\subsubsection{MongoDB}
MongoDB — це документо-орієнтована NoSQL (часто розшифровується як "not only SQL" - "не тільки SQL") база даних \cite{mongodb}. Її перевагами є гнучка схема, що дозволяє легко еволюціонувати структуру даних, горизонтальна масштабованість та вбудована підтримка геопросторових запитів (наприклад, у форматі GeoJSON), що є важливим для функціоналу карти вуликів та полів. Обрана для проекту завдяки гнучкості, підтримці GeoJSON та добрій інтеграції з Node.js через Mongoose.

\subsubsection{PostgreSQL}
PostgreSQL — потужна об'єктно-реляційна СУБД, відома своєю надійністю, відповідністю стандартам структурованої мови запитів (Structured Query Language, SQL) та розширюваністю (наприклад, PostGIS для геоданих). Хоча PostgreSQL є чудовим вибором для багатьох застосунків, для даного проекту з потенційно змінною структурою даних спільноти та геопросторовими об'єктами, гнучкість MongoDB була визнана більш пріоритетною.

\section{Обґрунтування вибору технологічного стеку}
\label{sec:tech_justification}
Вибір технологічного стеку для проекту веб-платформи для комунікації та обміну знаннями в спільноті бджолярів \textit{Beekeepers Community Platform} ґрунтувався на аналізі вимог, доступності інструментів, досвіді розробника та прагненні створити сучасний, масштабований та підтримуваний веб-застосунок. Було обрано наступний стек:
\begin{itemize}
    \item \textbf{Клієнтська частина (Frontend):} React \cite{react} з TypeScript, зібраний за допомогою Vite \cite{vite}, обрано через його популярність, велику екосистему, компонентний підхід та ефективність у створенні динамічних користувацьких інтерфейсів. Material-UI (MUI) \cite{materialui} використано як бібліотеку компонентів користувацького інтерфейсу (UI) для швидкої розробки адаптивного та естетично привабливого дизайну, що відповідає сучасним веб-стандартам. React Router використано для навігації, а Redux Toolkit (зокрема RTK Query) \cite{reduxtoolkit} — для управління станом та взаємодії з API.
    \item \textbf{Серверна частина (Backend):} NestJS (побудований на Node.js та Express/Fastify \cite{fastify}) з TypeScript обрано завдяки його модульній архітектурі, яка сприяє чіткій організації коду, вбудованій підтримці TypeScript, що підвищує надійність, та легкій інтеграції з іншими інструментами, такими як Passport.js \cite{passportjs} для автентифікації та Swagger (на основі OpenAPI \cite{openapi}) для автоматичної генерації документації API. Використання Fastify адаптера забезпечує високу продуктивність.
    \item \textbf{База даних:} MongoDB \cite{mongodb} обрано як NoSQL документо-орієнтовану базу даних. Її гнучка схема даних є перевагою для проекту, де структура інформації може розвиватися. Вбудована підтримка GeoJSON та геопросторових індексів є критично важливою для реалізації функціоналу інтерактивної карти. Mongoose \cite{mongoose} використано як об'єктно-документне відображення (Object Document Mapper, ODM) для взаємодії з MongoDB з боку NestJS.
    \item \textbf{Картографічний сервіс:} Leaflet разом з React-Leaflet обрано як легку та гнучку бібліотеку для відображення інтерактивних карт та маніпуляції геопросторовими даними (маркери, полігони) \cite{leaflet}, що відповідає сучасним тенденціям застосування геоінформаційних технологій для візуалізації даних в агросекторі \cite{granell2015agrogeoinformatics}.
    \item \textbf{Інтернаціоналізація:} i18next \cite{i18next} з react-i18next використано для підтримки багатомовності інтерфейсу.
    \item \textbf{Контейнеризація:} Docker \cite{docker} та Docker Compose використовуються для створення консистентного середовища розробки та спрощення розгортання застосунку.
    \item \textbf{Платформа розгортання та база даних у хмарі:} Для розгортання застосунку було обрано платформу як сервіс (Platform as a Service, PaaS) Render \cite{renderpaas} завдяки її простоті використання, інтеграції з GitHub для автоматичного розгортання (безперервна інтеграція/безперервне розгортання – Continuous Integration/Continuous Deployment, CI/CD), автоматичному налаштуванню захищеного протоколу передачі гіпертексту (Hypertext Transfer Protocol Secure, HTTPS) та можливості розгортання як статичних сайтів (фронтенд), так і Docker-контейнерів (бекенд). Для бази даних було обрано MongoDB Atlas \cite{mongodb} (як хмарний сервіс) через його переваги як керованої бази даних, що включають автоматичне резервне копіювання, масштабування та вбудовані засоби безпеки, що дозволило зосередитися на розробці застосунку, а не на адмініструванні інфраструктури.
\end{itemize}
Даний стек технологій дозволяє ефективно розробляти як клієнтську, так і серверну частини, забезпечуючи при цьому хорошу продуктивність, масштабованість та можливості для подальшого розвитку платформи. 