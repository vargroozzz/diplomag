\section*{Тези доповіді на наукову конференцію}
\addcontentsline{toc}{section}{Тези доповіді на наукову конференцію}
\label{sec:conference_abstract}

\subsection*{Вступ}
Бджільництво є невід'ємною частиною аграрного сектору та екосистеми, проте ефективна комунікація, обмін актуальними знаннями та координація дій між пасічниками часто ускладнені через відсутність спеціалізованих цифрових інструментів. Дана робота представляє \textit{Beekeepers Community Platform} – повностековий веб-застосунок, розроблений для задоволення потреб української спільноти бджолярів, шляхом надання єдиного середовища для обміну інформацією, управління пасіками та полями, а також доступу до профільної бази знань.

\subsection*{Методологія та технологічний стек}
Платформа розроблена з використанням сучасного стеку технологій, орієнтованого на продуктивність, масштабованість та зручність розробки. 
\textbf{Клієнтська частина (Frontend)} реалізована на React \cite{react} з використанням TypeScript, Vite \cite{vite} для збірки, Material-UI \cite{materialui} для користувацького інтерфейсу, Redux Toolkit (RTK Query) \cite{reduxtoolkit} для управління станом та взаємодії з API, і Leaflet \cite{leaflet} (з React-Leaflet) для картографічного функціоналу. 
\textbf{Серверна частина (Backend)} побудована на Node.js з фреймворком NestJS \cite{nestjs} (з адаптером Fastify \cite{fastify}) та TypeScript, що забезпечує модульну та типізовану архітектуру. Для автентифікації використовується Passport.js \cite{passportjs}.
\textbf{База даних} – MongoDB \cite{mongodb} (з Mongoose ODM \cite{mongoose}), обрана через гнучкість схеми та ефективну роботу з GeoJSON даними.
API розроблено за принципами RESTful з версіонуванням та документується за допомогою OpenAPI \cite{openapi}. Проєкт також включає інтернаціоналізацію (i18next \cite{i18next}) та підготовлений до контейнеризації (Docker \cite{docker}).

\subsection*{Реалізований функціонал}
Ключовими реалізованими модулями платформи є:
\begin{itemize}
    \item \textbf{Система автентифікації та управління користувачами:} Реєстрація з верифікацією електронної пошти (включаючи повторне надсилання листа), безпечний вхід з використанням JWT (access та refresh токени), можливість входу через Google OAuth. Користувачі мають можливість переглядати та оновлювати свої профілі.
    \item \textbf{Форум для обговорень:} Функціонал для створення тем, публікації повідомлень, коментування та система вподобань (лайків), що сприяє активній взаємодії в спільноті. (Деталі реалізації можуть бути розширені залежно від поточного стану розробки цього модуля).
    \item \textbf{Інтерактивна карта (Управління пасіками та полями):} Центральний компонент платформи, що дозволяє:
        \begin{itemize}
            \item Додавати, переглядати та видаляти (з підтвердженням) маркери вуликів, використовуючи кастомні іконки (MUI \texttt{HiveIcon}) для кращої візуалізації.
            \item Додавати та редагувати метадані полігональних об'єктів полів (назва, тип культури, період цвітіння, дати обробки).
            \item Динамічно візуалізувати статус обробки полів за допомогою зміни кольору полігонів (наприклад, червоний – обробка сьогодні, помаранчевий – обробка протягом найближчих 7 днів).
            \item Переглядати детальну інформацію про об'єкти у спливаючих вікнах (Popups) на карті.
        \end{itemize}
    \item \textbf{База знань:} Передбачено модуль для доступу до статей та ресурсів (наразі може бути на етапі прототипу з mock-даними, що варто зазначити).
\end{itemize}

\subsection*{Ключові технічні рішення та аспекти реалізації}
В процесі розробки було реалізовано низку цікавих технічних рішень:
\begin{itemize}
    \item Для відображення кастомних іконок вуликів (MUI \texttt{HiveIcon}) на карті Leaflet було використано поєднання \texttt{ReactDOMServer.renderToString} та \texttt{L.divIcon}, що дозволило інтегрувати React-компоненти в не-React середовище Leaflet.
    \item Реалізовано клієнтську логіку в компоненті \texttt{MapPage.tsx} для динамічного розрахунку та зміни кольору полігонів полів на основі дат їх обробки, що покращує інформативність карти.
    \item Для управління станом API, включаючи кешування, оптимістичні оновлення та обробку завантаження/помилок для CRUD операцій з вуликами та полями, активно використовується RTK Query.
    \item Забезпечено надійну доставку транзакційних email (верифікація, тощо) через інтеграцію з Mailgun та налаштуванням відповідних DNS записів (SPF, DKIM, DMARC).
\end{itemize}

\subsection*{Результати та обговорення}
Розроблений веб-застосунок \textit{Beekeepers Community Platform} є функціональним прототипом, що успішно реалізує заявлені ключові можливості. Платформа надає зручний інтерфейс та необхідні інструменти для бджолярів, сприяючи обміну досвідом та покращенню координації. Особливу цінність становить інтерактивна карта з можливістю управління геопросторовими даними пасік та полів, а також візуалізацією потенційно небезпечних періодів обробки полів.

\subsection*{Висновки та напрямки подальшого розвитку}
Створена платформа демонструє життєздатність концепції та має значний потенціал для подальшого розвитку. Першочерговими напрямками є розширення функціоналу форуму та бази знань, впровадження системи сповіщень (зокрема, про обробку полів поблизу пасік), розробка мобільного застосунку для підвищення доступності, а також інтеграція з іншими сервісами, корисними для бджолярів (наприклад, погодні сервіси, аналітика медозбору).

\textbf{Ключові слова:} веб-застосунок, спільнота бджолярів, React, NestJS, MongoDB, форум, інтерактивна карта, GeoJSON, автентифікація, Leaflet, RTK Query, управління пасіками.

\newpage 