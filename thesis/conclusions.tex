% --- Conclusions ---
\chapter*{Висновки}
\addcontentsline{toc}{chapter}{Висновки}
\label{ch:conclusions}

% TODO: Summarize the work done.
% TODO: Reiterate how the objectives were met.
% TODO: Discuss the main results and contributions of the thesis.
% TODO: Outline recommendations for future work and potential improvements.

У даній магістерській роботі було розроблено повностековий веб-додаток – веб-платформу для комунікації та обміну знаннями в спільноті бджолярів \textit{Beekeepers Community Platform}. 
Метою роботи було створення платформи для спілкування, обміну досвідом та знаннями серед бджолярів, а також надання інструментів для управління пасіками та полями.

Основні досягнуті результати:
\begin{itemize}
    \item Проведено аналіз предметної області та обґрунтовано вибір сучасного стеку технологій (React, NestJS, MongoDB).
    \item Спроектовано та реалізовано ключові функціональні модулі: система автентифікації (включаючи email верифікацію та Google OAuth 2.0), форум, база знань, інтерактивна карта.
    \item Забезпечено базові механізми безпеки та валідації даних.
    \item Створено адаптивний користувацький інтерфейс з використанням Material-UI.
\end{itemize}

Розроблений додаток успішно вирішує поставлені завдання, надаючи зручну та функціональну платформу для спільноти бджолярів. 

Напрямки для подальшого розвитку включають розширення функціоналу карти (фільтрація, аналітика), впровадження системи сповіщень, розробку мобільного додатку та інтеграцію з іншими сервісами для бджолярів. 