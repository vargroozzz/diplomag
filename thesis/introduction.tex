% --- Introduction ---
\chapter*{Вступ}
\addcontentsline{toc}{chapter}{Вступ}
\label{ch:introduction}

Актуальність теми магістерської роботи зумовлена зростаючою потребою у спеціалізованих онлайн-платформах для нішевих спільнот, зокрема для бджолярів, та визначається низкою ключових факторів. По-перше, бджільництво відіграє незамінну роль не лише як галузь сільського господарства, що забезпечує виробництво меду, воску, прополісу та інших цінних продуктів, але й як фундаментальний елемент підтримки біорізноманіття та екологічної стабільності через запилення ентомофільних культур та дикорослих рослин. За оцінками експертів, близько третини продовольства, що споживається людством, залежить від запилення комахами, серед яких бджоли є одними з найефективніших.

По-друге, сучасні бджолярі стикаються зі значними викликами, що загрожують як окремим пасікам, так і галузі в цілому. До таких викликів належать поширення хвороб та шкідників бджіл (наприклад, кліщ Варроа, американський гнилець), негативний вплив змін клімату на медоносні ресурси та життєдіяльність бджолосімей, а також масове використання пестицидів у сільському господарстві, що часто призводить до отруєння бджіл. В цих умовах оперативний обмін достовірною інформацією, передовим досвідом щодо профілактики та лікування хвороб, адаптації до кліматичних змін та безпечного співіснування з аграрним виробництвом набуває критичного значення.

По-третє, існуючі загальні соціальні мережі, месенджери та форуми, хоча й використовуються бджолярами для спілкування, не завжди враховують специфічні потреби та унікальний контекст бджолярської спільноти. Вони не надають спеціалізованих інструментів для обговорення вузькопрофільних тем, таких як ветеринарія бджіл, селекція, технології догляду за бджолосім\'ями, особливості медоносних рослин у конкретних регіонах. Особливо гострою є проблема координації дій щодо попередження отруєнь бджіл через обробку сільськогосподарських полів та ефективного планування розташування пасік з урахуванням кормової бази та санітарно-епідеміологічної ситуації.

Таким чином, розробка спеціалізованої веб-платформи, що інтегрує засоби комунікації, базу знань, інструменти для обміну оперативною інформацією (зокрема, про обробки полів) та можливості для ведення обліку пасік, є надзвичайно актуальним завданням. Така платформа може суттєво сприяти підвищенню ефективності бджільництва, збереженню бджолосімей, покращенню координації між пасічниками та іншими зацікавленими сторонами, а також слугувати інструментом для збору та аналізу даних, важливих для розвитку галузі та моніторингу екологічного стану.

Метою даної магістерської роботи є розробка фулстек веб-застосунку \textit{Beekeepers Community Platform}, що надасть бджолярам зручні інструменти для комунікації, обміну інформацією, доступу до бази знань та управління даними про власні пасіки та сільськогосподарські угіддя.

Для досягнення поставленої мети було визначено наступні завдання:
\begin{itemize}
    \item Провести детальний аналіз предметної області бджільництва, виявити ключові потреби цільової аудиторії та дослідити наявні на ринку аналоги та спеціалізовані програмні рішення для спільнот.
    \item Обґрунтувати вибір сучасного та ефективного технологічного стеку для розробки повнофункціонального веб-застосунку, враховуючи вимоги до масштабованості, продуктивності та зручності розробки клієнтської (React, TypeScript, Material-UI, Vite) та серверної (NestJS, Fastify, MongoDB) частин.
    \item Спроектувати комплексну архітектуру системи, включаючи:
        \begin{itemize}
            \item Архітектуру клієнтської частини з використанням компонентного підходу, управління станом (RTK Query, Context API) та адаптивного дизайну.
            \item Модульну архітектуру серверної частини на базі NestJS, що забезпечує чітке розділення функціональних блоків (автентифікація, управління користувачами, форум, карта, база знань).
            \item Логічну та фізичну схему бази даних MongoDB для ефективного зберігання та обробки користувацьких даних, геопросторової інформації, контенту форуму та бази знань.
        \end{itemize}
    \item Реалізувати ключовий функціонал платформи, зокрема:
        \begin{itemize}
            \item Систему реєстрації, автентифікації (локальна та через Google OAuth 2.0) та авторизації користувачів, включаючи механізм верифікації електронної пошти.
            \item Модуль форуму для створення тем, публікації повідомлень та коментування.
            \item Модуль інтерактивної карти (на базі Leaflet) для додавання, перегляду, редагування та видалення інформації про вулики та сільськогосподарські поля, з візуалізацією дат обробки полів.
            \item Прототип модуля бази знань з можливістю інтеграції AI-асистента для відповідей на поширені питання.
            \item Базовий функціонал адміністрування користувачів.
        \end{itemize}
    \item Імплементувати надійні механізми безпеки, включаючи хешування паролів, захист API-ендпоінтів (guards), валідацію вхідних даних на сервері (DTOs), та налаштування CORS.
    \item Розробити інтуїтивно зрозумілий, адаптивний та естетично привабливий користувацький інтерфейс, що забезпечує зручну взаємодію з платформою на різних пристроях, використовуючи компоненти Material-UI та принципи UX/UI дизайну.
    \item Провести функціональне тестування розробленого прототипу для перевірки коректності роботи основного функціоналу та підготувати платформу до можливого розгортання.
\end{itemize}

Об'єктом дослідження є процес проектування та розробки повностекового веб-застосунку для нішевої спільноти.

Предметом дослідження є архітектурні рішення, технології та інструменти для створення інтерактивної та функціональної платформи для бджолярів, що включає засоби комунікації, обміну знаннями та геоінформаційні функції.

Методи дослідження, що використовувалися в роботі, включають:
\begin{itemize}
    \item Аналіз науково-технічної літератури та існуючих аналогів.
    \item Системний аналіз та проектування програмного забезпечення.
    \item Об'єктно-орієнтоване програмування.
    \item Використання сучасних веб-технологій та фреймворків: React для розробки клієнтської частини, NestJS (Node.js) для серверної частини, MongoDB як система управління базами даних.
    \item Застосування бібліотек Material-UI для користувацького інтерфейсу та Leaflet для реалізації картографічного функціоналу.
\end{itemize}

\subsection*{Наукова новизна}
Наукова новизна одержаних результатів полягає у наступному:
\begin{itemize}
    \item Розроблено архітектуру та реалізовано функціональний прототип інтегрованої веб-платформи, що забезпечує синергетичне поєднання засобів соціальної комунікації (форум), спеціалізованої бази знань та інструментів геоінформаційного менеджменту (інтерактивна карта пасік та полів) для нішевої спільноти бджолярів. На відміну від розрізнених загальних інструментів, запропоноване комплексне рішення цілеспрямовано адаптоване до специфічних операційних та інформаційних потреб бджолярів України.
    \item Запропоновано та впроваджено оригінальний метод візуалізації на інтерактивній карті критично важливої для бджолярів інформації про заплановані дати та ризики обробки сільськогосподарських полів. Цей метод, що базується на динамічному кольоровому кодуванні полігонів відповідно до часової близькості обробок, є новим інструментом для підвищення ситуаційної обізнаності бджолярів та оперативного попередження потенційних загроз для бджолосімей у локальному контексті.
    \item Досліджено та обґрунтовано ефективність застосування сучасного, але збалансованого технологічного стеку (React, NestJS, Leaflet, MongoDB) для створення масштабованої та функціональної платформи, що задовольняє специфічні вимоги до розробки систем підтримки агро-геоінформаційних спільнот. Продемонстровано практичну реалізацію такого стеку для вирішення завдань у сфері бджільництва, спираючись на аналіз існуючих підходів до цифрової трансформації сільського господарства та управління онлайн-спільнотами \cite{preece2005onlinecommunities, huet2022digitalbeekeeping, guruprasad2024beeopen}.
    \item Розроблено та інтегровано прототип інтелектуального FAQ-асистента на основі великої мовної моделі (LLM), з адаптованим промпт-інжинірингом для надання контекстно-залежних відповідей на питання користувачів у специфічній галузі бджільництва, що є новим підходом до надання інформаційної підтримки в рамках подібних нішевих платформ.
\end{itemize}

Практичне значення отриманих результатів полягає у створенні готового до використання прототипу веб-платформи, що може бути впроваджена для підтримки спільноти бджолярів, покращення їх взаємодії та доступу до актуальної інформації.

\subsection*{Апробація результатів роботи}
% TODO: Fill this with actual conference presentations/publications or state if none.
Результати роботи були представлені у вигляді тез доповіді на конференції "Формування сучасної науки: методика та практика", 23.05.2025, м. Київ.
% Приклад: 
% Результати роботи були представлені у вигляді тез доповіді на XX Міжнародній науково-практичній конференції «Інформаційні технології в сучасному світі», 15-16 травня 2025 р., м. Київ.
% Публікація: [Ваше П.І.Б.]. Назва тез // Матеріали конференції... – С. XX-YY. (якщо є публікація)

Структура роботи: Магістерська робота складається зі вступу, чотирьох розділів, висновків, списку використаних джерел та додатків. 