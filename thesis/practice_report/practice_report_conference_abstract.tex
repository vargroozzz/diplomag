\section*{Тези доповіді на наукову конференцію}
\addcontentsline{toc}{section}{Тези доповіді на наукову конференцію}
\label{sec:conference_abstract}

Актуальність розробки спеціалізованих веб-платформ для нішевих спільнот зумовлена потребою в ефективному обміні знаннями та досвідом. Бджільництво, як важлива галузь, потребує таких інструментів для координації, поширення інформації про хвороби, методи догляду та медоносні культури. Існуючі загальні платформи не враховують специфіку цієї галузі.

Метою даної роботи є представлення веб-застосунку \textit{Beekeepers Community Platform} — фулстек рішення, розробленого для української спільноти бджолярів. Платформа надає інструменти для комунікації (форум), обміну знаннями (база знань) та управління даними про пасіки та поля (інтерактивна карта).

Технологічний стек включає React (Vite, TypeScript, Material-UI, RTK Query, Leaflet) для клієнтської частини та NestJS (Node.js, Fastify, TypeScript, Mongoose) для серверної, з MongoDB як основною базою даних. Архітектура побудована на принципах модульності та RESTful API. Реалізовано систему автентифікації (локальна, Google OAuth) з верифікацією email, форум з можливістю створення тем та лайків, інтерактивну карту з додаванням маркерів (вулики) та полігонів (поля) з метаданими (GeoJSON). Передбачено інтернаціоналізацію (українська/англійська).

Застосунок \textit{Beekeepers Community Platform} демонструє потенціал для покращення взаємодії та інформаційного забезпечення бджолярів, сприяючи розвитку галузі. Подальший розвиток може включати розширення аналітичних функцій, мобільний застосунок та інтеграцію з іншими сервісами.

\textbf{Ключові слова:} веб-застосунок, спільнота бджолярів, React, NestJS, MongoDB, форум, інтерактивна карта, GeoJSON, автентифікація.
\newpage 