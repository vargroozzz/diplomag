% --- Abstract in Ukrainian ---
\begin{otherlanguage}{ukrainian}
\begin{abstract}
\noindent

Магістерська робота присвячена проектуванню та розробці повностекового веб-застосунку \textit{Beekeepers Community Platform} — інтегрованої платформи, призначеної для української спільноти бджолярів з метою сприяння ефективній комунікації, обміну спеціалізованими знаннями та надання інструментів для управління пасіками і полями.

У роботі здійснено аналіз предметної області, розглянуто існуючі рішення для нішевих онлайн-спільнот та інструменти для агросектору. Обґрунтовано вибір сучасного технологічного стеку, що включає React (з TypeScript, Vite, Material-UI, Redux Toolkit) для розробки клієнтської частини, NestJS (Node.js, Fastify, TypeScript) для серверної логіки, та MongoDB (з Mongoose) як документо-орієнтовану базу даних з підтримкою GeoJSON.

Спроектовано архітектуру платформи для комунікації та обміну знаннями в спільноті бджолярів, що базується на принципах модульності та RESTful API. Визначено функціональні та нефункціональні вимоги. Розроблено ключові модулі: система автентифікації користувачів (локальна реєстрація, верифікація email з можливістю повторного надсилання листа, Google OAuth, JWT для авторизації), форум для обговорень, структурована база знань, та багатофункціональна інтерактивна карта. Картографічний модуль дозволяє користувачам додавати, переглядати, редагувати метадані та видаляти об'єкти пасік (вулики) і полів, а також візуалізує попередження про заплановані обробки полів шляхом динамічного кольорового кодування.

Описано процес розробки, організацію кодової бази, ключові аспекти реалізації API та інтерфейсу користувача, включаючи використання Leaflet для картографії. Розглянуто питання безпеки, валідації даних та інтернаціоналізації.

Проведено тестування основних функціональних можливостей. За результатами роботи сформульовано висновки щодо успішної реалізації прототипу платформи та окреслено перспективні напрямки для її подальшого розвитку та вдосконалення.

\textbf{Ключові слова:} веб-застосунок, спільнота бджолярів, React, NestJS, MongoDB, інтерактивна карта, Leaflet, GeoJSON, управління пасіками, автентифікація, форум, база знань, RTK Query, Material-UI.
\end{abstract}
\end{otherlanguage} 