% Numbering: 1., 2., ... for main; 1.1., 1.2., ... for subitems
\renewcommand{\theenumi}{\arabic{enumi}}
\renewcommand{\labelenumi}{\theenumi.}
\renewcommand{\theenumii}{\arabic{enumi}.\arabic{enumii}}
\renewcommand{\labelenumii}{\theenumii.}

\begin{center}
    \section*{\bfseries\MakeUppercase{Питання з охороною праці при роботі з комп'ютером, принтером, ксероксом та іншою оргтехнікою}}
\end{center}
\addcontentsline{toc}{section}{Питання з охороною праці при роботі з комп'ютером, принтером, ксероксом та іншою оргтехнікою}

\begin{enumerate}
    \item Загальні вимоги безпеки при роботі з комп'ютером та іншою оргтехнікою
    \begin{enumerate}
        \item До самостійної роботи з комп'ютером, ноутбуком, принтером, ксероксом, сканером, плазмовою панеллю, LCD-дисплеєм та іншою оргтехнікою допускаються особи, які досягли 18 річного віку, пройшли медичний огляд, ознайомлені з інструкцією з охорони праці при роботі з оргтехнікою, не мають протипоказань за станом здоров'я.
        \item Під час роботи на комп'ютері та іншій оргтехніці на вчителя можуть впливати наступні небезпечні та шкідливі фактори:
        \begin{itemize}
            \item електрострум і випромінювання;
            \item перенапруження зору під час роботи з електронними пристроями, монітором, особливо при нераціональному розташуванні екрана по відношенню до очей.
        \end{itemize}
        \item Освітлювальні установки повинні забезпечувати рівномірне освітлення і не повинні утворювати засліплюючих відблисків на клавіатурі, а також на екрані монітора за напрямом очей.
        \item При роботі з комп'ютером, принтером, ксероксом та іншою периферійною технікою не допускається розташування робочого місця в приміщеннях без природного освітлення, без наявності природної або штучної вентиляції.
        \item Робоче місце з комп'ютером та оргтехнікою повинно розміщуватися на відстані не менше 1м від стіни, від стіни з віконними отворами - на відстані не менше 1,5 м.
        \item Кут нахилу екрана монітора або ноутбука по відношенню до вертикалі повинен складати 10-15 градусів, а відстань до екрана - 500-600 мм.
        \item Кут зору екрана повинен бути прямим і становити 90 градусів.
        \item Для захисту від прямих сонячних променів повинні передбачатися сонцезахисні пристрої (плівка з металізованим покриттям, регульовані жалюзі з вертикальними панелями та ін).
        \item Освітлення повинно бути змішаним (природним та штучним).
        \item У приміщенні кабінету і на робочому місці необхідно підтримувати чистоту і порядок, проводити систематичне провітрювання.
        \item Про всі виявлені під час роботи несправності обладнання необхідно доповісти керівнику, у випадку поломки необхідно припинити роботу до усунення аварійних обставин. При виявленні можливої небезпеки, попередити оточуючих та негайно повідомити керівнику; утримувати в чистоті робоче місце, не захаращувати його сторонніми предметами.
        \item Про нещасний випадок очевидець, працівник, який його виявив, або сам потерпілий повинні доповісти безпосередньо керівникові установи і вжити заходів з надання медичної допомоги.
        \item Особи, винні в порушенні вимог, вимагаємих даною інструкцією з охорони праці при роботі з комп'ютером, принтером, ксероксом та іншою оргтехнікою, притягаються до дисциплінарної відповідальності у відповідності з чинним законодавством.
    \end{enumerate}
    \item Вимоги безпеки перед початком роботи з комп'ютером (ноутбуком) та іншою оргтехнікою
    \begin{enumerate}
        \item Оглянути і переконатися у справності обладнання, електропроводки. У разі виявлення несправностей, до роботи не приступати. Повідомити про це керівника і, тільки після усунення несправностей і його дозволу, приступити до роботи.
        \item Перевірити освітлення робочого місця, за необхідності, вжити заходів до його нормалізації.
        \item Перевірити наявність та надійність захисного заземлення устаткування.
        \item Перевірити стан електричного шнура і вилки.
        \item Перевірити справність вимикачів та інших органів управління персональним комп'ютером та оргтехніки.
        \item При виявленні будь-яких несправностей, комп'ютер та оргтехніку не вмикати і негайно повідомити про це завідувача дошкільним навчальним закладом.
        \item Ретельно провітрити приміщення з персональним комп'ютером та оргтехнікою, переконатися, що мікроклімат у приміщенні знаходиться в допустимих межах: температура повітря в холодний період року 22-24°С, в теплий період року - 23-25°С, відносна вологість повітря — 40-60%.
        \item Включити монітор і перевірити стабільність і чіткість зображення на екрані, переконатися у відсутності запаху диму від комп'ютера та оргтехніки.
    \end{enumerate}
    \item Вимоги безпеки під час роботи з комп'ютером, ноутбуком, принтером, ксероксом, сканером, плазмовою панеллю, LCD-дисплеєм та іншою оргтехнікою
    \begin{enumerate}
        \item Вмикайте і вимикайте комп'ютер, ноутбук та іншу оргтехніку тільки вимикачами, забороняється проводити вимкнення витягуванням вилки з розетки.
        \item Забороняється знімати захисні пристрої з обладнання і працювати без них.
        \item Не допускати до комп'ютера та оргтехніки сторонніх осіб, які не беруть участі в роботі.
        \item Забороняється переміщати та переносити системний блок, монітор, принтер, будь-яке обладнання, яке знаходиться під напругою.
        \item Забороняється під час роботи пити будь-які напої, приймати їжу.
        \item Забороняється будь-яке фізичне втручання у пристрій комп'ютера, принтера, сканера, ксерокса під час їх роботи.
        \item Забороняється залишати включене обладнання без нагляду.
        \item Забороняється класти предмети на комп'ютерне обладнання, монітори, екрани та оргтехніку.
        \item Суворо виконувати загальні вимоги з електробезпеки та пожежної безпеки.
        \item Під час усунення застрявання паперу на ксероксі чи принтері, задля уникнення ураження електрострумом, необхідно відключити обладнання від електромережі. Необхідно також вимикати обладнання від мережі при тривалому простої.
        \item Самостійно розбирати та проводити ремонт електронної та електронномеханічної частини комп'ютера, периферійних пристроїв, оргтехніки категорично забороняється. Ці роботи може виконувати тільки спеціаліст або інженер з технічного обслуговуваннюя комп'ютерної техніки.
        \item Сумарний час безпосередньої роботи з персональним комп'ютером та іншою оргтехнікою протягом робочого дня має бути не більше 6 годин, для педагогів — не більше 4 годин у день.
        \item Тривалість безперервної роботи з персональним комп'ютером та іншою оргтехнікою без регламентованої перерви не повинна перевищувати 2-х годин. Через кожну годину роботи слід робити перерву тривалістю 15 хв.
        \item Під час регламентованих перерв, з метою зниження нервово-емоційного напруження, стомлення зорового аналізатора, усунення впливу гіподинамії та гіпокінезії, запобігання розвитку познотонічного стомлення, слід виконувати комплекси вправ для очей або організовувати фізкультурні паузи.
        \item Комп'ютер, будь-які його периферійні пристрої, оргтехніку необхідно використовувати у суворій відповідності з експлуатаційною документацією до них.
        \item Під час виконання роботи необхідно бути уважним, не звертати уваги на сторонні речі.
        \item Про всі виявлені несправності та збої в роботі апаратури необхідно повідомити безпосередньо інженера з обслуговування комп'ютерної техніки або керівника практики.
    \end{enumerate}
    \item Вимоги безпеки після закінчення роботи з комп'ютером, принтером, ксероксом, сканером та іншою оргтехнікою
    \begin{enumerate}
        \item Вимкнути комп'ютер, ноутбук, телевізор, плазмову панель, LCDекран, принтер, ксерокс, сканер, колонки та іншу оргтехніку від електромережі, для чого необхідно вимкнути тумблери, а потім акуратно витягнути штепсельні вилки з розетки.
        \item Протерти зовнішню поверхню комп'ютера чистою вологою тканиною. При цьому не допускайте використання розчинників, одеколону, препаратів в аерозольній упаковці.
        \item Прибрати робоче місце. Скласти диски у відповідне місце зберігання.
        \item Ретельно провітрити приміщення з персональним комп'ютером та іншою оргтехнікою.
    \end{enumerate}
    \item Вимоги техніки безпеки та безпеки життєдіяльності в аварійних ситуаціях при роботі з комп'ютером та іншою оргтехнікою
    \begin{enumerate}
        \item Якщо на металевих частинах обладнання виявлено напругу (відчуття струму), заземлюючий провід обірваний, необхідно вимкнути обладнання, негайно доповісти керівникові про несправності електрообладнання і без його вказівки до роботи не приступати.
        \item При припиненні подавання електроенергії, вимкнути обладнання.
        \item При появі незвичного звуку, запаху паленого, мимовільного відключення комп'ютера та оргтехніки, негайно припинити роботу і поставити до відома керівника.
        \item При виникненні пожежі негайно вимкнути обладнання, знеструмити електромережу за винятком освітлювальної мережі, повідомити про пожежу всім працюючим і приступити до гасіння осередку пожежі наявними засобами пожежогасіння.
        \item При нещасному випадку необхідно, насамперед, звільнити потерпілого від травмуючого фактора, звернутися до медпункту, зберегти, по можливості, місце травмування в тому стані, в якому воно було на момент травмування. При звільненні потерпілого від дії електроструму слідкуйте за тим, щоб самому не опинитися в контакті з токоведучою частиною та під напругою.
    \end{enumerate}
\end{enumerate} 